\documentclass[11pt, a4paper]{beamer}
\usepackage{amsmath}
\usepackage{amsfonts}
\usepackage{amssymb}
\usepackage{tikz}

\usetheme{Frankfurt}
\usecolortheme{seagull}
\usepackage{xltxtra,fontspec}
\usepackage{polyglossia}
\setmainlanguage{english}
\defaultfontfeatures{Scale=MatchLowercase}

\usepackage[absolute,overlay]{textpos}

\setbeamertemplate{footline}
{
  \leavevmode%
  \hbox{%
  \begin{beamercolorbox}[wd=.6\paperwidth,ht=2.25ex,dp=1ex,center]{author in head/foot}%
    \usebeamerfont{author in head/foot}\insertshortauthor
  \end{beamercolorbox}%
  \begin{beamercolorbox}[wd=.4\paperwidth,ht=2.25ex,dp=1ex,center]{title in head/foot}%
    \usebeamerfont{title in head/foot}\insertshorttitle\hspace*{3em}
    \insertframenumber{} / \inserttotalframenumber\hspace*{1ex}
  \end{beamercolorbox}}%
  \vskip0pt%
}


\setbeamertemplate{navigation symbols}{}

\author{Dan Häberlein, Peggy Lucke, J. Nathanael Philipp, Alexander Richter}
\title{openLegislature}
\date{}
\institute{Universität Leipzig}

%\logo{\includegraphics[scale=0.25]{./LGD_Logo.png}}
%\setbeamersize{text margin left=7mm, text margin right=7mm}

%\usepackage[backend=bibtex8,style=numeric,subentry]{biblatex}
%\usepackage[babel]{csquotes}
%\defbibheading{bibliography}{}
%\bibliography{quellen}

\begin{document}
\section{}
\begin{frame}
\titlepage
\end{frame}

%\only<presentation| handout:0> {
%	\AtBeginSection[]{%
%		\begin{frame}
%		\frametitle{Gliederung}
%		\tableofcontents[currentsection]
%		\end{frame}
%	}% AtBeginSection
%}

%\only<2| handout:1>{
	\begin{frame}
		\frametitle{Outline}
		\tableofcontents
	\end{frame}
%}

\section{Introduction}
\begin{frame}
\frametitle{Introduction}
\begin{itemize}
\item Korpus
\end{itemize}
\end{frame}

\subsection{Korpus}
\begin{frame}{Informationen}
\textbf{Plenarprotokolle des Bundestages}
\begin{itemize}
\item stenographische Berichte
\item ab erste Sitzung des Bundestages September 1949
\item vollständig bis zur letzten abgeschlossenen Sitzung
\item PDF-Format
\item aktuelle Wahlperiode auch als Textdatei
\end{itemize}
\end{frame}

\begin{frame}
\textbf{Zugang:}
\begin{itemize}
\item öffentlich
\item frei zugänglich
\item http://suche.bundestag.de/plenarprotokolle/search.form\\[1em]
\end{itemize}
\textbf{Download per Funktionen von:}
\begin{itemize}
\item \textit{Muster:}\\
http://dip21.bundestag.de/dip21/btp/ [Wahlperiode]/[Wahlperiode][Sitzung].pdf
\item \textit{Bsp.:}\\ http://dip21.bundestag.de/dip21/btp/01/01029.pdf\\[1em]
\end{itemize}
\textbf{Größe des Korpus}
\begin{itemize}
\item Momentan 3895 PDF-Dateien
\item Entspricht ca. 10GB
\end{itemize}
\end{frame}

\section{Methods}
\begin{frame}
\frametitle{Methods}
\begin{itemize}
\item 
\end{itemize}
\end{frame}

\section{Outlook}
\begin{frame}
\frametitle{Outlook}
\begin{itemize}
\item Nächste Schritte
\item Ziele Vorlesungszeit
\item Ziele Semester
\item Erweiterbarkeit
\item Einschränkungen
\end{itemize}
\end{frame}

\frame{
\frametitle{Ausblick I: nächste Schritte}

Nächste Schritte:
\begin{itemize}
	\item  erweitern der Metadaten-Datenbank mit Metadaten:
		\begin{itemize}
			\item aus bestehenden XML Files
			\item aus zusätzlichen Quellen (z.B. Sitzverteilungen)
	 	\end{itemize}
	\item  Überführen der Strukturierten Daten im XML-Format in einen Document Store
	\item  Analyse der Daten: 
		\begin{itemize}
			\item Clustering (Top-Down, Bottom-Up)
			\item LDA (Latend Dirichlet Allocation)
	 	\end{itemize}
\end{itemize}
}


\frame{
\frametitle{Ausblick II: Ziele Vorlesungszeit}

Ziele bis Ende Vorlesungszeit:
\begin{itemize}
	\item XML-Daten in Document-Store ablegen
	\item Metadaten-Datenbank mit weiteren Metadaten erweitern
	\item analysieren der Daten mittels mind. zwei Clustering-Verfahren 
	\item Cluster mit wahrscheinlich gleichen Schreiber (aber nicht Redner) finden und darstellen
\end{itemize}
}

\frame{
\frametitle{Ausblick III: Ziele Semester}

Ziele bis Ende Semester:
\begin{itemize}
	\item Analyse mittels LDA
	\item Visualisierung der Ergebnisse der LDA-Analysen
	\item weitere Cluster-Verfahren nutzen
	\item alle (sinnvollen) Ergebnisse vereinen und darstellen
	\item Untersuchung warum manche Analysen fehlerhafte/schlechte Ergebnisse lieferten
\end{itemize}
}


\frame{
\frametitle{Ausblick IV: Erweiterbarkeit}

Erweiterbarkeit wenn uns Zeit bleibt:
\begin{itemize}
	\item zusätzlichen Metadaten-Quellen auffinden und in die bestehende Metadaten-Datenbank überführen
	\item weitere Metadaten erzeugen (Bsp. POS-Tagging, N-Gramme und Kookkurrenzen)
	\item Anaylse-Verfahren erweitern 
	\begin{itemize}
		\item andere Cluster-Algorithmen
		\item LDA mit anderen Parametern
		\item LDA mit anderen Features
	\end{itemize}
	\item Visualisieren der Ergebnisse
\end{itemize}
}


\frame{
\frametitle{Ausblick V: Einschränkungen I}

Einschränkungen wenn wir nicht alle Ziele schaffen:
\begin{itemize}
	\item weniger Analyseverfahren nutzen (Bsp. nur ein Clusteringverfahren)
	\item Datenbereinigung verkürzen
	\item weniger Metadaten als Feature nutzen
\end{itemize}
}


\frame{
\frametitle{Ausblick VI: Einschränkungen II}

Wichtigste Ziele:
\begin{itemize}
	\item Daten in Datenbank strukturiert ablegen
	\item Clustering-Verfahren auf unsere Daten anwenden
	\item Interpretation der Ergebnisse
\end{itemize}

Neue Ziele:
\begin{itemize}
	\item vermutlich nicht
\end{itemize}
}

\section{Give us your money!}
\begin{frame}
\frametitle{Give us your money!}
We try to achieve reproducable and professional results. Our project could be
really interessting in the following sence:
\begin{itemize}
\item History / Political Science  
\item Educational Purposes
\item Parties 
\end{itemize}
We could also make this dataset that we just created more human accessible by 
developping an easy user interface (something like google).
Our work would contribute to more transparent german politics, in which every 
citizen has the power to validate and measure politicians by there speeches. 
\end{frame}
\end{document}
