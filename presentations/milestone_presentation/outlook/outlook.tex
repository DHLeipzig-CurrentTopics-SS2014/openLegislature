\documentclass{beamer}
\usepackage[utf8]{inputenc}
\usepackage{graphicx}
\usepackage{ngerman}


\begin{document}

\frame{
\frametitle{Ausblick I: nächste Schritte}

Nächste Schritte:
\begin{itemize}
	\item  ertweitern der Metadaten-Datenbank mit Metadaten:
		\begin{itemize}
			\item aus bestehenden XML files
			\item aus zusätzlichen Quellen (z.B. Sitzverteilungen)
	 	\end{itemize}
	\item  Überführen der Strukturierten Daten im XML-Format in einen Document Store
	\item  Analyse der Daten: 
		\begin{itemize}
			\item Clustering (Top-Down, Bottom-up)
			\item LDA (Latend Dirichlet Allocation)
	 	\end{itemize}
\end{itemize}
}


\frame{
\frametitle{Ausblick II: Ziele Vorlesungszeit}

Ziele bis Ende Vorlesungszeit:
\begin{itemize}
	\item XML-Daten in Document-Store ablegen
	\item Metadaten-Bank mit weiteren Metadaten erweitern
	\item analysieren der Daten mittels mind. zwei Clustering-Verfahren, 
	\item Cluster mit wahrscheinlich gleichen Schreiber (aber nicht Redner) finden und darstellen
\end{itemize}
}

\frame{
\frametitle{Ausblick III: Ziele Semester}

Ziele bis Ende Semester:
\begin{itemize}
	\item Analyse mittels LDA
	\item Visualisierung der Ergebnisse der LDA-Analysen
	\item weitere Cluster-Verfahren nutzen
	\item alle (sinnvollen) Ergebnisse vereinen und Darstellen
	\item Untersuchung warum manche Analysen fehlerhafte/schlechte Ergebnisse lieferten
\end{itemize}
}


\frame{
\frametitle{Ausblick IV: Erweiterbarkeit}

Erweiterbarkeit wenn uns Zeit bleibt:
\begin{itemize}
	\item zusätzlichen Metadaten-Quellen auffinden und in die bestehende Metadaten-Datenbank überführen
	\item weitere Metadaten erzeugen zB. Pos-tagging, n-gramme und Kookkurrenzen
	\item Anaylse-Verfahren erweitern 
	\begin{itemize}
		\item andere Cluster-Algorithmen
		\item LDA mit anderen Parametern
		\item LDA mit anderen Features
	\end{itemize}
	\item Visualisieren der Ergebnisse
\end{itemize}
}


\frame{
\frametitle{Ausblick V: Einschränkungen I}

Einschränkungen wenn wir nicht alle Ziele schaffen:
\begin{itemize}
	\item weniger Analyseverfahren nutzen, zB. nur ein Clusteringverfahren
	\item Datenbereinigung verkürzen
	\item weniger Metadata als feature nutzen
\end{itemize}
}


\frame{
\frametitle{Ausblick VI: Einschränkungen II}

Wichtigste Ziele:
\begin{itemize}
	\item Daten in Datenbank strukturiert ablegen
	\item Clustering-Verfahren auf unsere Daten Anwenden
	\item Interpretation der Ergebnisse
\end{itemize}

neue Ziele:
\begin{itemize}
	\item vermutlich nicht
\end{itemize}
}


\end{document}
