\documentclass[11pt, a4paper]{beamer}
\usepackage{amsmath}
\usepackage{amsfonts}
\usepackage{amssymb}
\usepackage{tikz}
\usetheme{Frankfurt}
\usecolortheme{seagull}
\usepackage{xltxtra,fontspec}
\usepackage{polyglossia}
\setmainlanguage{english}
\defaultfontfeatures{Scale=MatchLowercase}

\usepackage[absolute,overlay]{textpos}

\setbeamertemplate{footline}
{
  \leavevmode%
  \hbox{%
  \begin{beamercolorbox}[wd=.6\paperwidth,ht=2.25ex,dp=1ex,center]{author in head/foot}%
    \usebeamerfont{author in head/foot}\insertshortauthor
  \end{beamercolorbox}%
  \begin{beamercolorbox}[wd=.4\paperwidth,ht=2.25ex,dp=1ex,center]{title in head/foot}%
    \usebeamerfont{title in head/foot}\insertshorttitle\hspace*{3em}
    \insertframenumber{} / \inserttotalframenumber\hspace*{1ex}
  \end{beamercolorbox}}%
  \vskip0pt%
}


\setbeamertemplate{navigation symbols}{}

\author{Dan Häberlein, Peggy Lucke, J. Nathanael Philipp, Alexander Richter}
\title[The openLegislature project]{Current Topics in Digital Philology\\The openLegislature project}
\date{}
\institute{Universität Leipzig}

%\logo{\includegraphics[scale=0.25]{./LGD_Logo.png}}
%\setbeamersize{text margin left=7mm, text margin right=7mm}

%\usepackage[babel]{csquotes}
%\defbibheading{bibliography}{}
%\bibliography{quellen}

\usepackage[backend=biber,style=numeric]{biblatex}
\addbibresource{lit.bib}

\begin{document}
\section{}
\begin{frame}
\titlepage
\end{frame}

\only<presentation| handout:0> {
	\AtBeginSection[]{%
		\begin{frame}
		\frametitle{Gliederung}
		\tableofcontents[currentsection]
		\end{frame}
	}% AtBeginSection
}

\only<2| handout:1>{
	\begin{frame}
		\frametitle{Outline}
		\tableofcontents
	\end{frame}
}

\section{Introduction}
\subsection{Korpus}
\begin{frame}{Informations}
\textbf{Plenary Protocols from Bundestag}
\begin{itemize}
\item stenographic reports in PDF
\item open to the public
\item siehe \cite{bundestag} \\[1em]
\item size of corpus circa 10GB $\rightarrow$more than 3900 PDF 
\end{itemize}
\end{frame}

\subsection{Questions}
\begin{frame}{Questions to the information in the corpus}
\textbf{Statistic:}
\begin{itemize}
\item How many speakers are in one legislative period/total?
\item How many speeches from one party/speaker?
\end{itemize}
\textbf{Keyword-search:}
\begin{itemize}
\item Which speaker spoke to a special topic?
\end{itemize}
\textbf{Why this questions?}\\
We want more transparency! The answers are there, but too difficult to reach for all other people. That will be changed!
\end{frame}
\section{Methoden}
\subsection{Vorgehensweise}
\begin{frame}
\frametitle{Methoden 1}
\textbf{Preprocessing:}
\begin{itemize}
\item Lemmatizing
\item ggfs. Part of Speech Analyse  
\item Vergleichen von N-Grammen zur Ähnlichkeitsüberprüfung
\item Keine Stopwortentfernung, da dadurch Informationen verloren gehen!
\end{itemize}
\end{frame}

\begin{frame}
\frametitle{Methoden 2}
\textbf{IR Methoden:}
\begin{itemize}
\item Kookkurrenz auf verschiedenen Ebenen (Reden per Partei, Redner, Gesammt)
\item Clustering
\end{itemize}
\textbf{IR Algorithmen}
\begin{itemize}
\item Ähnlichkeitsmaße berechen und vergleichen 
\item Topic Modell / LDA 
\end{itemize}
\textit {"Latent Dirichlet allocation (LDA) is a generative probabilistic
	model of a corpus. The basic idea is that documents are represented as
 	random mixtures over latent topics, where each topic is characterized by a
 	distribution over words."}, siehe \cite{blatent}
%  	
\end{frame} 

\begin{frame}
  \frametitle{Architektur Prozess Datenextraktion und -aufbereitung }
  \begin{itemize}
  \item Nutzung des Listener Patterns \cite{javainsel9}
  \item Verwendung der Github-Library Async \cite{async} zur einfachen Erstellung Nebenläufiger Prozessketten
  \end{itemize}  
  \begin{center}
    \includegraphics[width=1\textwidth]{../../doc/process-overview.png}
  \end{center}
\end{frame}


\section{Results}
\begin{frame}
\frametitle{temporary results}
\begin{itemize}
\item unstructured Textfiles avaiable (PDF / TXT) for all election periods
\item semi-structured XML files processed from PDF
\item Metadatabase with data of all election periods
\item XPath query's on XML-Files
\end{itemize}
\end{frame}

\begin{frame}
\frametitle{temporary results 2}
\begin{itemize}
\item NoSQL Database:
	\begin{itemize}
		\item all speeches
		\item Speaker with all Speeches
		\item appearance of words by speech
		\item Speakerstatistics over all election periods
		\item Partystatistics over all election periods
	\end{itemize}
\item website for browsing corpusdata/-statistics with visualisations
\item imput files for slda (arff-files) generated ( for 18th election period )
\item slda output: significant words for each speech of the 18th election period 
\end{itemize}
\end{frame}

\begin{frame}
\frametitle{Statistics}
\begin{itemize}
\item 18 election periods
\item 7004 speaker
\item 679910 speeches
\item 39 partys
\item But: data not as clean as possible
\begin{itemize}
\item typing errors: e.g. "CSU/CSU", "Angelika Me rtens", ..
\item parsing problems
\end{itemize}
\end{itemize}
\end{frame}

\begin{frame}
\frametitle{Statistics 2: Speeches pro election period}
\includegraphics[scale=0.4]{speechperperiod.png}
\end{frame}

\begin{frame}
\frametitle{significant words for speeches}
18th election period, second session: Thomas Oppermann
\begin{itemize}
\item 1. staat
\item 2. verhandeln
\item 3. snowden
\item 4. nsa
\item 5. praxis
\item 6. geheimdienste
\item 7. ausspioniert
\item 8. hören
\item 9. möglichkeit
\item 10. schutz
\end{itemize}
\end{frame}

\begin{frame}
\frametitle{significant words for speeches 2}
18th election period, third session: Oskar Lafontaine
\begin{itemize}
\item 1. waffenexporte
\item 2. währung
\item 3. ökonomisch
\item 4. zukunftsaufgaben
\item 5. währungsspekulation
\item 6. übernachtungen
\item 7. verteilung
\item 8. waggons
\item 9. zug
\item 10. schneller
\end{itemize}
\end{frame}

\section{Outlook}

\subsection{next Steps}
\begin{frame}
\frametitle{Outlook I: next Steps}
Next Steps:
\begin{itemize}
	\item Clustering (Top-Down, Bottom-Up)
	\item LDA (Latend Dirichlet Allocation)
\end{itemize}
\end{frame}

\subsection{Ziele}
\begin{frame}
\frametitle{Ausblick II: Ziele Vorlesungszeit}

Ziele bis Ende Vorlesungszeit:
\begin{itemize}
	\item XML-Daten in Document-Store ablegen
	\item Metadaten-Datenbank mit weiteren Metadaten erweitern
	\item analysieren der Daten mittels mind. zwei Clustering-Verfahren 
	\item Cluster mit wahrscheinlich gleichen Schreiber (aber nicht Redner) finden und darstellen
\end{itemize}
\end{frame}

\begin{frame}
\frametitle{Ausblick III: Ziele Semester}

Ziele bis Ende Semester:
\begin{itemize}
	\item Analyse mittels LDA
	\item Visualisierung der Ergebnisse der LDA-Analysen
	\item weitere Cluster-Verfahren nutzen
	\item alle (sinnvollen) Ergebnisse vereinen und darstellen
	\item Untersuchung warum manche Analysen fehlerhafte/schlechte Ergebnisse lieferten
\end{itemize}
\end{frame}

\subsection{Projektumfang}
\begin{frame}
\frametitle{Ausblick IV: Erweiterbarkeit}

Erweiterbarkeit wenn uns Zeit bleibt:
\begin{itemize}
	\item zusätzlichen Metadaten-Quellen auffinden und in die bestehende Metadaten-Datenbank überführen
	\item weitere Metadaten erzeugen (Bsp. POS-Tagging, N-Gramme und Kookkurrenzen)
	\item Anaylse-Verfahren erweitern 
	\begin{itemize}
		\item andere Cluster-Algorithmen
		\item LDA mit anderen Parametern
		\item LDA mit anderen Features
	\end{itemize}
	\item Visualisieren der Ergebnisse
\end{itemize}
\end{frame}


\begin{frame}
\frametitle{Ausblick V: Einschränkungen I}

Einschränkungen wenn wir nicht alle Ziele schaffen:
\begin{itemize}
	\item weniger Analyseverfahren nutzen (Bsp. nur ein Clusteringverfahren)
	\item Datenbereinigung verkürzen
	\item weniger Metadaten als Feature nutzen
\end{itemize}
\end{frame}


\begin{frame}
\frametitle{Ausblick VI: Einschränkungen II}

Wichtigste Ziele:
\begin{itemize}
	\item Daten in Datenbank strukturiert ablegen
	\item Clustering-Verfahren auf unsere Daten anwenden
	\item Interpretation der Ergebnisse
\end{itemize}

Neue Ziele:
\begin{itemize}
	\item Einfaches Query Interface (ähnlich Google) um Nutzern Zugang zu Daten zu
	geben
	\item Query Interface als Webanwendung 
\end{itemize}
\end{frame}

\section{Give us your money!}
\begin{frame}
\frametitle{Give us your money!}
We try to achieve reproducable and professional results. Our project could be
really interessting in the following sence:
\begin{itemize}
\item History / Political Science  
\item Educational Purposes
\item Parties 
\end{itemize}
We could also make this dataset that we just created more human accessible by 
developping an easy user interface (something like google).
Our work would contribute to more transparent german politics, in which every 
citizen has the power to validate and measure politicians by there speeches.
\end{frame}

\printbibliography

\end{document}
